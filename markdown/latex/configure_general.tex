{[}\href{basicComponents.html}{Previous: Terrier Components}{]}
{[}\href{index.html}{Contents}{]}
{[}\href{configure_indexing.html}{Next: Configuring Indexing}{]}

\section{Configuring Terrier}\label{configuring-terrier}

\section{Configuring Terrier}\label{configuring-terrier-1}

\subsection{Configuring Overview}\label{configuring-overview}

Terrier is configured overall by a few files, all in the \texttt{etc/}
directory. The most central files are \texttt{terrier.properties} and
\texttt{terrier-log.xml}. In \texttt{terrier.properties}, you can
specify any of the various properties that are defined in Terrier. The
\href{properties.html}{Properties} documentation page lists the most
used properties that you need to configure Terrier, while the
\href{javadoc/}{javadoc} for any class lists the properties that
directly affect the class. The default \texttt{terrier.properties} file
is given below:

\begin{verbatim}

#default controls for query expansion
querying.postprocesses.order=QueryExpansion
querying.postprocesses.controls=qe:QueryExpansion

#default and allowed controls
querying.default.controls=
querying.allowed.controls=qe,start,end,qemodel

#document tags specification
#for processing the contents of
#the documents, ignoring DOCHDR
TrecDocTags.doctag=DOC
TrecDocTags.idtag=DOCNO
TrecDocTags.skip=DOCHDR

#query tags specification
TrecQueryTags.doctag=TOP
TrecQueryTags.idtag=NUM
TrecQueryTags.process=TOP,NUM,TITLE
TrecQueryTags.skip=DESC,NARR

#stop-words file
stopwords.filename=stopword-list.txt

#the processing stages a term goes through
termpipelines=Stopwords,PorterStemmer
\end{verbatim}

In the terrier.properties file, properties are specified in the format
\texttt{name=value} (the default Java Properties format). Comments are
lines starting with \texttt{\#}.

\subsubsection{Scripting Properties}\label{scripting-properties}

TrecTerrier supports specifying properties on the command line. This
allows the easy over-riding of properties, even if they are specified in
the \texttt{etc/terrier.properties} file. For example, to create an
index without using a stemmer, you could use the command line:

\begin{verbatim}
[user@machine]$ bin/trec_terrier.sh -i -Dtermpipelines=Stopwords
\end{verbatim}

Aside: When looking for properties, Terrier also checks the
\href{http://download.oracle.com/javase/tutorial/essential/environment/sysprop.html}{System
properties provided by Java}. This means that you can set a property
anywhere within Java code, or on the Java command line.

As another example, you can use shell scripting (e.g. Bash) to run
Terrier with many settings for the \texttt{expansion.terms} property of
query expansion:

\begin{verbatim}
[user@machine]$ for((i=2;i<10;i++)); do 
    bin/trec_terrier.sh -r -q -Dexpansion.terms=$i 
done
\end{verbatim}

\subsubsection{Configuring Logging}\label{configuring-logging}

Terrier uses \href{http://logging.apache.org/log4j/1.2/}{Log4j} for
logging. You can control the amount of logging information (the logging
level) that Terrier outputs by altering the log4j config in
\texttt{etc/terrier-log.xml}. For more information about configuring
Log4j, see the
\href{http://logging.apache.org/log4j/1.2/manual.html}{Log4j
documentation}.

{[}\href{basicComponents.html}{Previous: Terrier Components}{]}
{[}\href{index.html}{Contents}{]}
{[}\href{configure_indexing.html}{Next: Configuring Indexing}{]}

\begin{center}\rule{0.5\linewidth}{\linethickness}\end{center}

Webpage: \url{http://terrier.org}\\
Contact:
\href{mailto:terrier@dcs.gla.ac.uk}{\nolinkurl{terrier@dcs.gla.ac.uk}}\\
\href{http://www.dcs.gla.ac.uk/}{School of Computing Science}\\
Copyright (C) 2004-2015 \href{http://www.gla.ac.uk/}{University of
Glasgow}. All Rights Reserved.
