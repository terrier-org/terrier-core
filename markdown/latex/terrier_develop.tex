{[}\href{realtime_indices.html}{Previous: Real-time Index Structures}{]}
{[}\href{index.html}{Contents}{]} {[}\href{extend_indexing.html}{Next:
Extending Indexing}{]}\\

\section{Developing Applications with
Terrier}\label{developing-applications-with-terrier}

Terrier provides APIs for \href{extend_indexing.html}{indexing}
documents, and \href{extend_retrieval.html}{querying} the generated
indices. If you are developing applications using Terrier or extending
it for your own research, then you may find the following information
useful.

\subsection{Extending Terrier}\label{extending-terrier}

Terrier has a very flexible and modular architecture, with many classes,
some with various alternatives. It is very easy to change many parts of
the indexing and retrieval process. Essential to any in-depth extension
of Terrier is to examine the very many
\href{properties.html}{properties} that can be configured in Terrier.
For instance, if you write a new Matching class, you can use this in a
TREC-like setting by setting the property \texttt{trec.matching}, while
if you write a new document weighting model you should set the property
\texttt{trec.model} to use it. For more information about extending the
retrieval functionalities of Terrier, see
\href{extend_retrieval.html}{Extending Retrieval}, and
\href{extend_indexing.html}{Extending Indexing} for more information
about the indexing process Terrier uses.

\subsubsection{FileSystem Abstraction
Layer}\label{filesystem-abstraction-layer}

All File IO in Terrier (excluding the Desktop application and Terrier
configuration) is performed using the
\href{javadoc/org/terrier/utility/Files.html}{Files} class. This affords
various opportunities for allowing Terrier to run in various
environments. In Terrier, a FileSystem abstraction layer was integrated
into the Files class, such that other
\href{javadoc/org/terrier/utility/io/FileSystem.html}{FileSystem}
implementations could be plugged in. By default, Terrier ships with two
implementation, namely
\href{javadoc/org/terrier/utility/io/LocalFileSystem.html}{LocalFileSystem}
for reading the local file system using the Java API, and
\href{javadoc/org/terrier/utility/io/HTTPFileSystem.html}{HTTPFileSystem}
for reading files accessible by HTTP or HTTPS protocols. A filename is
searched for a prefixing scheme (eg ``file://''), similar to a URI or
URL. If a scheme is detected, then Terrier will search through its known
file system implementations for a file system supporting the found
scheme. file:// is the default scheme if no scheme can be found in the
filename; if the filename starts http://, then the file will be fetched
by HTTP. Since Terrier 2.2, this abstraction layer has also supported
Hadoop Distributed Filesystem for prefixes with hdfs:// - for more
information, see \href{hadoop_configuration.html}{Configuring Terrier
for Hadoop}.

The Files layer can also transform paths to filenames on the fly. For
example, if a certain HTTP namespace is accessible as a local file
system, the Files layer can be informed using
\texttt{Files.addPathTransformation()}. If you have a slow network file
system, consider using the in-built caching layer in Files.

Additional implementations can implement methods of the FileSystem
interface that they support, and register themselves by calling the
\texttt{Files.addFileSystemCapability()} method. The FileSystem denotes
the operations it supports on a file or path by returning the bit-wise
OR of the constants named in Files.FSCapability.

Finally, the Files layer supports transparent compression for reading or
writing file streams. In particular, compressed files using Gzip (.gz)
and Bzip2 (.bz2) can be obtained by just adding the extension to the
file. Moreover, alternative compression/decompression libraries can be
added using the \texttt{addFilterInputStreamMapping()} method.

\href{}{}

\subsection{Compiling Terrier}\label{compiling-terrier}

The main Terrier distribution comes pre-compiled as Java, and can be run
on any Java 1.7 JDK. You should have no need to compile Terrier unless
you have altered the Terrier source code and wish to check or use your
changes.

Terrier now uses \href{https://maven.apache.org}{Maven} for
dependencies, compiling testing and packaging. Terrier's layout does not
yet follow the Maven standard layout, with various source folders
located under src/. Finally, the Maven environment is configured to
build with Eclipse also, although a few plugins are disabled.

The following Maven goals can be used for recompiling Terrier:

\begin{itemize}
\tightlist
\item
  \texttt{mvn\ compile} - compile
\item
  \texttt{mvn\ package} - make the jar and jar-with-dependencies, as
  well as the .tar.gz and .zip files
\item
  \texttt{mvn\ package\ -DskipTests} - as above, but skipping the JUnit
  tests
\end{itemize}

\subsection{Testing Terrier}\label{testing-terrier}

Terrier now has many JUnit test classes, located into the
\texttt{src/test} folder. In particular, JUnit tests are now provided
for a great many of the classes in Terrier, including (but not limited
to) indexers, tokenisation, retrieval, query parsing, compression, and
evaluation.

In addition, there are JUnit-based end-to-end tests that ensure that the
expected results are obtained from a small (22 document) corpus
consisting of Shakespeare's play, the Merchant of Venice. The end-to-end
tests test all indexers, as well as retrieval functionality behaves as
expected. The corpus, test topics and relevance assesments are located
in \texttt{share/tests/shakespeare}. Running the unit and Shakespeare
end-to-end tests takes about 5 minutes, and can be performed from the
command line using the Ant \texttt{test} target.

\protect\hyperlink{wt2g}{}

New in Terrier 4.0 is an end-to-end test based on the TREC WT2G corpus.
This is good method test to ensure that a change on Terrier has not
negatively impacted on retrieval performance. Note that while indexing
the WT2G corpus only takes a few minutes, the end-to-end test suite runs
different indexing varieties, so you should allow about about an hour
for this test to run. To run this test, you need to specify the location
of the WT2G corpus, topics and relevance assessments (qrels):

\begin{verbatim}
bin/anyclass.sh -Dwt2g.corpus=/path/to/WT2G/ -Dwt2g.topics=/path/to/WT2G_topics/small_web/topics.401-450 -Dwt2g.qrels=/path/to/WT2G_topics/small_web/qrels.trec8 org.junit.runner.JUnitCore org.terrier.tests.TRECWT2GEndtoEndTest
\end{verbatim}

{[}\href{realtime_indices.html}{Previous: Real-time Index Structures}{]}
{[}\href{index.html}{Contents}{]} {[}\href{extend_indexing.html}{Next:
Extending Indexing}{]}

\begin{center}\rule{0.5\linewidth}{\linethickness}\end{center}

Webpage: \url{http://terrier.org}\\
Contact:
\href{mailto:terrier@dcs.gla.ac.uk}{\nolinkurl{terrier@dcs.gla.ac.uk}}\\
\href{http://www.dcs.gla.ac.uk/}{School of Computing Science}\\
Copyright (C) 2004-2015 \href{http://www.gla.ac.uk/}{University of
Glasgow}. All Rights Reserved.
