{[}\href{evaluation.html}{Previous: Evaluation of Experiments}{]}
{[}\href{index.html}{Contents}{]} {[}\href{terrier_http.html}{Next:
Web-based Terrier}{]}\\

\section{Using the Desktop Search example
application:}\label{using-the-desktop-search-example-application}

Desktop Terrier is an example application we have provided with Terrier
for two purposes:

\begin{itemize}
\tightlist
\item
  To provide a Desktop Search application that will allow users to
  quickly test out features of Terrier such as for example the Terrier
  query language.
\item
  To give developers an example of using Terrier in an interactive
  setting.
\end{itemize}

Importantly, Desktop Terrier is only a sample application to help users
become used to the functionality that Terrier provides. We do not
recommend Desktop Terrier to perform large or complex indexing jobs.
Instead, once you are comfortable with the Terrier functionality,
indexing and batch retrieval should be performed using the command line.
You have been warned.

\subsection{Starting Desktop Terrier}\label{starting-desktop-terrier}

\begin{itemize}
\tightlist
\item
  \textbf{Windows}: Double click on
  bin\textbackslash{}desktop\_terrier.bat to start Desktop Terrier - on
  some versions of Windows you may receive a warning about the file
  being suspicious, but you can safely ignore this. Note that you will
  need at least Java 1.7 or later to run Desktop Terrier.
\item
  \textbf{MacOS X}: Double click on bin/desktop\_terrier.sh to start
  Desktop Terrier. Should this fail:

  \begin{enumerate}
  \tightlist
  \item
    Make sure you have Java 1.7
  \item
    Select bin/desktop\_terrier.sh in Finder
  \item
    In File menu, select Get Info (Command-I)
  \item
    Select ``Terminal'' application with ``Open with''. Terminal is in
    the Folder Applications/Utilities.
  \end{enumerate}
\item
  \textbf{Unix/Linux}: execute the bin/desktop\_terrier.sh shell script
  to start Desktop Terrier. You can do this from an Xterm environment
  (or similar), or by double clicking bin/desktop\_terrier.sh in a
  Konqueror or Nautilus window (KDE or Gnome).
\end{itemize}

\subsection{Running Desktop Terrier}\label{running-desktop-terrier}

The application window of the Desktop Search features two main tabs:
``Search'' and ``Index''. In the following, we will explain how you can
use the application to index and search documents on your computer.

\subsubsection{Indexing}\label{indexing}

Here we will explain how you can specify which documents you want
Desktop Terrier to index.

Indexing is the process where Terrier examines all the files in the
folders you specified, reads the documents if it can, and creates an
index. There are only two buttons on the ``Index'' tab. The ``Select
Folders\ldots{}'' button opens a dialog that will allow you to select
which folders should be indexed. The application will examine these
folders recursively, and will index all the supported document types.
Based on the file extension, the application will try to find a
corresponding parser. If no appropriate parser can be found, the file
will be ignored. At the moment Terrier supports parsing of Simple text,
PDF, MS Word, MS PowerPoint, MS Excel, HTML, XML, XHTML, Tex, and Bib
documents. Importantly, Desktop Terrier uses SimpleFileCollection, hence
each file counts as a single document. More complex formats like those
used at TREC are not detected by default. We recommend using Terrier
from the command line to process these types of collection.

The ``Create Index'' button will initiate the indexing process. At the
moment the Desktop Terrier does not support the updatable indices. That
means that every time you press the ``Create Index'' button Terrier will
remove the old index and index all specified folders from scratch. Once
you have selected the folders to index, you may click the ``Create
Index'' button in order to start the indexing process. The progress of
the indexing is documented in the text field at the bottom of the
window, After the indexing has finished, the application will
automatically switch to the ``Search tab''.

You can now use the Search tab of Desktop Terrier to search for
documents. Enter terms that you think your document may contain in the
text box beside the Search button, and press Search. Documents Terrier
thinks are relevant will be displayed in the list below. You can open a
document by double clicking on that row in the table. The type of the
document is shown in the second column.

\subsubsection{Searching}\label{searching}

In the searching tab, you can enter a query in the text field and press
the button ``Search'' to obtain the retrieval results. The results are
shown in the table below the search field, as a ranked list of
documents. The table has four columns. The first one contains the rank
of a document, the second one contains the file name of a document. The
third one contains the full path to the document and finally the fourth
one contains the score of the document.

To formulate a query, you can incorporate the
\href{querylanguage.html}{query language of Terrier}. For example:

\begin{itemize}
\tightlist
\item
  The query \texttt{"information\ retrieval"} should retrieve documents
  where the two query terms are either in the same, or in consecutive
  blocks.
\item
  The query \texttt{"information\ retrieval"\textasciitilde{}5} should
  retrieve documents in which the query terms appear within 5 blocks,
  irrespectively of the query term order.
\item
  With the operators plus or minus, we may specify that a term should,
  or shouldn't appear in the retrieved documents. For example, for the
  query \texttt{information\ retrieval\ +book}, the retrieved documents
  should at least contain the term book.
\item
  In the query \texttt{information\ retrieval\^{}2.5}, the query term
  \texttt{retrieval} has a 2.5 times higher weight than what it would
  have normally.
\item
  The query \texttt{information\ retrieval\ c:7} will perform retrieval
  for the query terms \texttt{information} and \texttt{retrieval},
  setting the value of the term frequency normalisation parameter
  \texttt{c} equal to 7.
\end{itemize}

By default, Terrier Desktop Search retrieves the documents that contain
all the query terms. If there are no such documents, then it returns the
documents that contain at least one of the query terms.

In order to open one of the retrieved documents, you may double-click on
its filename, i.e. the corresponding cell of the second column. Opening
the retrieved files is a platform-dependent function. In Windows and Mac
OS X environments, the application uses the file associations used by
the operating system, while in other environments, such as Linux the
file associations need to be set by the user. In these cases, the
associations are saved in a file with the default name desktop.fileassoc
in the var directory of your installation.

If there is already an application associated with the file, then this
application will start and open the file you double-clicked on. In the
case when there is no application associated, a dialog will appear, in
order to assist you with selecting an appropriate application.

\subsection{Help}\label{help}

This documentation is also available from the Help menu of the Desktop
Terrier version.

\subsection{Advanced Options}\label{advanced-options}

Should you have trouble using Desktop Terrier, e.g. if the application
is not running as expected, you can make use of the ``--debug'' option:

\begin{verbatim}
bin/desktop_terrier.sh --debug (Linux, Mac OS X)
bin\desktop_terrier.bat --debug (Windows)
\end{verbatim}

If you use Desktop Terrier regularly, you may wish to have Terrier
re-index your documents automatically at set times. You can do this by
scheduling Terrier to run with the ``--reindex'' option:

\begin{verbatim}
bin/desktop_terrier.sh --reindex (Linux, Mac OS X)
bin\desktop_terrier.bat --reindex (Windows)
\end{verbatim}

In order to schedule this command line for repetitive execution on Unix
use the crontab utility. On Windows use the Scheduled Tasks
functionality, which can be found in the Control Panel.

\subsection{Advanced Configuration}\label{advanced-configuration}

You can configure the Desktop using many of the properties listed
elsewhere in the Terrier documentation. These can be set in the
\texttt{etc/terrier.properties} file. Moreover, it is possible to
configure the Desktop using the following properties:

\textbf{Properties:}

\begin{itemize}
\tightlist
\item
  \texttt{desktop.directories.spec} - Where is the collection.spec for
  the desktop. Defaults to \texttt{var/desktop.spec}
\item
  \texttt{desktop.matching} - Which matching class to use for desktop.
  Defaults to org.terrier.matching.taat.Full.
\item
  \texttt{desktop.model} - Which weighting model to use for the desktop.
  Defaults to PL2.
\item
  \texttt{desktop.manager} - Which Manager class to use for the desktop.
  Defaults to Manager.
\item
  \texttt{desktop.indexing.singlepass} - Set to true to use the
  single-pass indexer.
\item
  \texttt{desktopsearch.filetype.colors} - Mapping of file type to
  colour. Default value
  \texttt{Text:(221\ 221\ 221),TeX:(221\ 221\ 221),Bib:(221\ 221\ 221),PDF:(236\ 67\ 69),HTML:(177\ 228\ 250),Word:(100\ 100\ 255),Powerpoint:(250\ 110\ 49),Excel:(38\ 183\ 78),XHTML:(177\ 228\ 250),XML:(177\ 228\ 250)}
\item
  \texttt{desktopsearch.filetype.types} - Comma-delimited mapping of
  file extensions to File types. Default value is
  \texttt{txt:Text,text:Text,tex:TeX,bib:Bib,pdf:PDF,html:HTML,htm:HTML,xhtml:XHTML,xml:XML,doc:Word,ppt:Powerpoint,xls:Excel}
\end{itemize}

{[}\href{evaluation.html}{Previous: Evaluation of Experiments}{]}
{[}\href{index.html}{Contents}{]} {[}\href{trec_examples.html}{Next:
TREC Experiment Examples}{]}

\begin{center}\rule{0.5\linewidth}{\linethickness}\end{center}

Webpage: \url{http://terrier.org}\\
Contact:
\href{mailto:terrier@dcs.gla.ac.uk}{\nolinkurl{terrier@dcs.gla.ac.uk}}\\
\href{http://www.dcs.gla.ac.uk/}{School of Computing Science}\\
Copyright (C) 2004-2015 \href{http://www.gla.ac.uk/}{University of
Glasgow}. All Rights Reserved.
