{[}\href{evaluation.html}{Previous: Evaluation}{]}
{[}\href{index.html}{Contents}{]} {[}\href{terrier_desktop.html}{Next:
Desktop Search in Terrier}{]}\\

\section{Real-time Indexing with
Terrier}\label{real-time-indexing-with-terrier}

\subsection{Introduction}\label{introduction}

In addition to traditional on-disk indices, Terrier provides both
memory-only and hybrid memory+disk index structures that can be updated
dynamically with new documents over time. Since Terrier 4.0, the top
level \href{javadoc/org/terrier/structures/Index.html}{Index} class
became abstract such that different types of indices can be supported.
The pre-Terrier 4.0 index functionality is contained within the
\href{javadoc/org/terrier/structures/IndexOnDisk.html}{IndexOnDisk}
class, while new index types were added to enable search systems that
can be updated in real-time without a lengthy batch indexing process.

\subsection{Index Interfaces}\label{index-interfaces}

To support real-time indexing, two new interfaces have been defined,
namely
\href{javadoc/org/terrier/realtime/UpdatableIndex.html}{UpdatableIndex}
and
\href{javadoc/org/terrier/realtime/WritableIndex.html}{WritableIndex}.
An index class that implements WritableIndex supports the dynamic
addition of new documents via a indexDocument() method. When
indexDocument() is called, that document will be added to the index
immediately and will be searchable once the indexDocument() returns. The
WritableIndex interface represents an index that can be written to disk.
In particular, a class that implements WritableIndex will implement a
write() method that will convert each of the index structures into
equivalent on-disk structures and will be written out to a specified
path and with a named prefix. An index written in this way can then be
later loaded as an IndexOnDisk index.

\subsection{Real-time Index Types}\label{real-time-index-types}

There are two real-time index structures supported in Terrier 4.0:

\begin{itemize}
\tightlist
\item
  \href{javadoc/org/terrier/realtime/memory/MemoryIndex.html}{MemoryIndex}:
  Represents an index that is held wholly in memory. MemoryIndex is both
  an UpdatableIndex and a WritableIndex. MemoryIndex is designed to
  provide a fast updatable index structure for relatively small numbers
  of documents.
\item
  \href{javadoc/org/terrier/realtime/incremental/IncrementalIndex.html}{IncrementalIndex}:
  A hybrid index structure that combines a MemoryIndex with zero or more
  IndexOnDisk indices, facilitating the updating of a large index that
  could not be stored in memory alone. An incremental index is a
  \href{javadoc/org/terrier/realtime/multi/MultiIndex.html}{MultiIndex},
  where one index shard is stored in memory and the rest are stored on
  disk. Periodically, the memory index is then written to disk, defined
  as per a FlushPolicy. When the memory index has been flushed to disk,
  optionally the on-disk portion of the incremental index can then be
  merged together (based upon a MergePolicy) and/or deleted (based upon
  a DeletePolicy). Incremental index uses the following properties:

  \begin{itemize}
  \tightlist
  \item
    incremental.flush: the flush policy to use. Four possible values are
    supported: noflush (default), flushdocs, flushmem, flushtime
  \end{itemize}

  \begin{itemize}
  \tightlist
  \item
    incremental.merge: the merge policy to use. Three possible values
    are supported: nomerge (default), single, geometric
  \end{itemize}

  \begin{itemize}
  \tightlist
  \item
    incremental.delete: the delete policy to use. Two possible values
    are supported: nodelete (default), deleteFixedSize
  \end{itemize}
\end{itemize}

\subsection{Usage}\label{usage}

Below we give some examples for using the real-time Terrier index
structures.

\begin{itemize}
\tightlist
\item
  We provide a Website search demo that illustrates a use of MemoryIndex
  to search over webpages as they are crawled. For more information
  about this demo see \href{website_search.html}{Real-time Indexing and
  Search of Websites}.
\item
  Any custom Java application can make use of an updatable index using
  MemoryIndex or IncrementalIndex, a code sample that illustrates
  indexing of a document and then search for that document is provided
  below:
\end{itemize}

\begin{verbatim}
// define an example document and query
String docContent = "Real-time indexing and retrieval is easy to use in Terrier";
String query = "Indexing";

// create a new index
MemoryIndex memIndex = new MemoryIndex();

// get the default tokeniser to break the document down into words
Tokeniser tokeniser = Tokeniser.Tokeniser.getTokeniser();

// create a Terrier document from the content string
Reader contentReader = new StringReader(docContent);
Map documentProperties = new HashMap();
FileDocument document = new FileDocument(contentReader, documentProperties, tokeniser);

// index the document
memIndex.indexDocument(document);

// the document is now available for searching

// create a search manager (runs the search process over an index)
Manager queryingManager = new Manager(memIndex);

// a search request represents the search to be carried out
SearchRequest srq = queryingManager.newSearchRequest("query", sb.toString());
srq.setOriginalQuery(sb.toString());

// define a matching model, in this case use the classical BM25 retrieval model
srq.addMatchingModel("Matching","BM25");

// run the four stages of a Terrier search
queryingManager.runPreProcessing(srq);
queryingManager.runMatching(srq);
queryingManager.runPostProcessing(srq);
queryingManager.runPostFilters(srq);

ResultSet results = srq.getResultSet();
\end{verbatim}

{[}\href{evaluation.html}{Previous: Evaluation}{]}
{[}\href{index.html}{Contents}{]} {[}\href{terrier_desktop.html}{Next:
Desktop Search in Terrier}{]}

\begin{center}\rule{0.5\linewidth}{\linethickness}\end{center}

Webpage: \url{http://terrier.org}\\
Contact:
\href{mailto:terrier@dcs.gla.ac.uk}{\nolinkurl{terrier@dcs.gla.ac.uk}}\\
\href{http://www.dcs.gla.ac.uk/}{School of Computing Science}\\
Copyright (C) 2004-2015 \href{http://www.gla.ac.uk/}{University of
Glasgow}. All Rights Reserved.

~
