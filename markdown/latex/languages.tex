{[}\href{compression.html}{Previous: Pluggable Compression}{]}
{[}\href{index.html}{Contents}{]} {[}\href{dfr_description.html}{Next:
DFR Description}{]}\\

\section{Non English language support in
Terrier}\label{non-english-language-support-in-terrier}

\subsection{Indexing}\label{indexing}

Terrier internally represents all terms as UTF. All provided Document
classes use
\href{javadoc/org/terrier/indexing/tokenisation/Tokeniser.html}{Tokeniser}
classes to tokenise text into terms during indexing. Likewise, during
retrieval,
\href{javadoc/org/terrier/structures/TRECQuery.html}{TRECQuery} uses the
same tokeniser to parse queries (note that, different from TRECQuery,
\href{javadoc/org/terrier/structures/SingleLineTRECQuery.html}{SingleLineTRECQuery}
does not perform any tokenisation by default). To change the tokeniser
being used for indexing and retrieval, set the \texttt{tokeniser}
property to the name of the tokeniser you wish to use (NB: British
English spelling). When indexing a batch collection using
\href{javadoc/org/terrier/indexing/TRECCollection.html}{TRECCollection},
the \href{javadoc/org/terrier/indexing/Document.html}{Document}
implementations should be informed of the expected character set using
the property \texttt{trec.encoding}.

\subsection{File Encodings}\label{file-encodings}

While Terrier uses UTF internally to represent terms, the Collection and
Document classes need to ensure that they are correctly opening files
using the correct character encodings. For instance, while valid XML
files will specify the encoding at the top of the file, a corpus of
Hungarian in TREC format may be encoded in ISO 8859-16 or UTF-8. You
should specify the encoding that TRECCollection should use to open the
files, using the \texttt{trec.encoding} property. Note that Terrier will
default to the Java default encoding if \texttt{trec.encoding} is not
set. In a Unix-like operating system, Java's choice of default encoding
may be influenced by the LANG environment variable - e.g.
LANG=en\_US.UTF-8 will cause Java to default to opening files using
UTF-8 encoding, while en\_US will use ISO-8859-1.

\subsection{Tokenisers}\label{tokenisers}

Tokenisers are designed to identify the terms from a stream of text. It
is expected that no markup will be present in the text passed to the
tokenisers (for indexing, the removal of markup is handled by
\href{javadoc/org/terrier/indexing/Document.html}{Document}
implementations - e.g. HTML tags are parsed by
\href{javadoc/org/terrier/indexing/TaggedDocument.html}{TaggedDocument}).
The choice of tokeniser to use depends on the language being dealt with.
Terrier 3.5 ships with three different tokenisers for use when indexing
text or parsing queries. The choice of tokeniser is specified by the
tokeniser property, e.g. \texttt{tokeniser=EnglishTokeniser}.

\begin{itemize}
\tightlist
\item
  \href{javadoc/org/terrier/indexing/tokenisation/EnglishTokeniser.html}{EnglishTokeniser}
  - assumes that all valid characters in terms are A-Z, a-z and 0-9.
  Obviously this assumption is incorrect when indexing documents in
  languages other than English.
\item
  \href{javadoc/org/terrier/indexing/tokenisation/UTFTokeniser.html}{UTFTokeniser}
  - uses Java's Character class to determine what valid characters in
  indexing terms are. In particular, a term can only contain characters
  matching one of Character.isLetterOrDigit(), Character.getType()
  returns Character.NON\_SPACING\_MARK or Character.getType() returns
  Character.COMBINING\_SPACING\_MARK.
\item
  \href{javadoc/org/terrier/indexing/tokenisation/IdentityTokeniser.html}{IdentityTokeniser}
  - a simple tokeniser that returns the input text as is, and is used
  internally by
  \href{javadoc/org/terrier/structures/SingleLineTRECQuery.html}{SingleLineTRECQuery}.
\end{itemize}

\subsubsection{Stemmers}\label{stemmers}

Terrier includes all stemmers from the
\href{http://snowball.tartarus.org/}{Snowball} stemmer project, namely:

\begin{itemize}
\tightlist
\item
  \href{javadoc/org/terrier/terms/DanishSnowballStemmer.html}{DanishSnowballStemmer}
\item
  \href{javadoc/org/terrier/terms/DutchSnowballStemmer.html}{DutchSnowballStemmer}
\item
  \href{javadoc/org/terrier/terms/EnglishSnowballStemmer.html}{EnglishSnowballStemmer}
\item
  \href{javadoc/org/terrier/terms/FinnishSnowballStemmer.html}{FinnishSnowballStemmer}
\item
  \href{javadoc/org/terrier/terms/FrenchSnowballStemmer.html}{FrenchSnowballStemmer}
\item
  \href{javadoc/org/terrier/terms/GermanSnowballStemmer.html}{GermanSnowballStemmer}
\item
  \href{javadoc/org/terrier/terms/HungarianSnowballStemmer.html}{HungarianSnowballStemmer}
\item
  \href{javadoc/org/terrier/terms/ItalianSnowballStemmer.html}{ItalianSnowballStemmer}
\item
  \href{javadoc/org/terrier/terms/NorwegianSnowballStemmer.html}{NorwegianSnowballStemmer}
\item
  \href{javadoc/org/terrier/terms/PortugueseSnowballStemmer.html}{PortugueseSnowballStemmer}
\item
  \href{javadoc/org/terrier/terms/RomanianSnowballStemmer.html}{RomanianSnowballStemmer}
\item
  \href{javadoc/org/terrier/terms/RussianSnowballStemmer.html}{RussianSnowballStemmer}
\item
  \href{javadoc/org/terrier/terms/SpanishSnowballStemmer.html}{SpanishSnowballStemmer}
\item
  \href{javadoc/org/terrier/terms/SwedishSnowballStemmer.html}{SwedishSnowballStemmer}
\item
  \href{javadoc/org/terrier/terms/TurkishSnowballStemmer.html}{TurkishSnowballStemmer}
\end{itemize}

\subsection{Batch Retrieval}\label{batch-retrieval}

When experimenting with topics in files other than English, use the same
\texttt{tokeniser} setting used during indexing. Moreover, you should
also use the property \texttt{trec.encoding} to ensure that the correct
encoding is used when reading the topic files.

{[}\href{compression.html}{Previous: Pluggable Compression}{]}
{[}\href{index.html}{Contents}{]} {[}\href{dfr_description.html}{Next:
DFR Description}{]}

\begin{center}\rule{0.5\linewidth}{\linethickness}\end{center}

Webpage: \url{http://terrier.org}\\
Contact:
\href{mailto:terrier@dcs.gla.ac.uk}{\nolinkurl{terrier@dcs.gla.ac.uk}}\\
\href{http://www.dcs.gla.ac.uk/}{School of Computing Science}\\
Copyright (C) 2004-2015 \href{http://www.gla.ac.uk/}{University of
Glasgow}. All Rights Reserved.
